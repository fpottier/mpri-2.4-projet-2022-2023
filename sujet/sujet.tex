\documentclass{article}
\usepackage{amssymb}
\usepackage{amstext}
\usepackage{xcolor}
\usepackage{enumitem}
\setlist{nosep}
\usepackage[T1]{fontenc}
\usepackage[utf8]{inputenc}
\usepackage[a4paper,bottom=40mm]{geometry}
\usepackage[colorlinks=true,linkcolor=blue,citecolor=blue,urlcolor=blue]{hyperref}
\usepackage{macros}
\usepackage{times}
\usepackage[scaled=0.78]{beramono}
\usepackage{xspace}
\usepackage[final]{listings}
% Configuring listings for OCaml.

% Comments in blue.
\newcommand{\ocamlcommentstyle}{\color{blue}}

\lstdefinelanguage{ocaml}[Objective]{Caml}{
  % Fix errors in the default definition of ocaml.
  deletekeywords={closed,ref},
  morekeywords={initializer,effect,perform,continue},
  % General settings.
  flexiblecolumns=false,
  showstringspaces=false,
  framesep=5pt,
  commentstyle=\ocamlcommentstyle,
  % By default, we use a small font.
  basicstyle=\tt\small,
  numberstyle=\footnotesize,
  % LaTeX escape.
  mathescape=true,
  % LaTeX escape as an OCaml comment.
  escapeinside={(*@}{*)},
  % Range markers.
  rangeprefix=(*\ ,% opening comment plus space
  rangesuffix=\ *),% space plus closing comment
}

% An abbreviation for \lstinline, with a normal font size.
% Should work both in text mode and in math mode.
\def\oc|#1|{\text{\lstinline[language=ocaml,basicstyle=\tt,flexiblecolumns=true]|#1|}}

% With quotes.
\def\qoc|#1|{``\oc|#1|''}

% An environment.
\lstnewenvironment{ocaml}{\lstset{language=ocaml}}{}

\lstset{
  language=ocaml,
  aboveskip=.3\baselineskip,
  belowskip=.3\baselineskip,
}
\usepackage{mdframed}
\usepackage{natbib}
\bibliographystyle{plainnat}
\citestyle{authoryear}

\title{MPRI 2.4 \\
       Programmation fonctionnelle et systèmes de types \\
       Programming project}
\author{François Pottier}
\date{2022--2023}

\parskip=1mm

\begin{document}
\maketitle

% ------------------------------------------------------------------------------

\section{Overview}

The purpose of this project is to implement (selected parts of) a small
compiler which performs \textbf{forward-mode and reverse-mode automatic
  differentiation} for a very small language of arithmetic expressions.

A good introduction to automatic differentiation (AD) can be found in the
textbook by \citet{griewank-walther}. Chapters 1--3 are sufficient. Please
email \href{mailto:francois.pottier@inria.fr}{François Pottier} for an
electronic copy (not for redistribution).

The project is based on the paper ``You Only Linearize Once: Tangents
Transpose to Gradients''~\citep{radul-al-23}, which will be presented
at the conference POPL 2023 in January 2023.
%
The paper defines a calculus named \textbf{Linear} and presents three
transformations of this calculus into itself:
\begin{itemize}
\item forward-mode AD (Section~5);
\item unzipping (Section~6);
\item transposition (Section~7).
\end{itemize}
The composition of the three transformations, in the order shown above,
yields reverse-mode AD.

The project is organized in \textbf{two main tasks}. Task~1 is to implement
forward-mode AD. Task~2 is to implement unzipping and transposition, thereby
yielding an implementation of reverse-mode AD. A~test infrastructure is
provided for each of these two implementations of AD.

% ------------------------------------------------------------------------------

\section{The Surface calculus}

Surface is a simple calculus of arithmetic expressions. It is equipped with
functions and tuples. The syntax of expressions is as follows:
%
\[\begin{array}{rcll}
\uop & ::= &
  \cos \mid \sin \mid \exp \mid \ldots
  & \text{unary operations}
\\
\bop & ::= &
  \mathord{+} \mid \mathord{-} \mid \mathord{\times} \mid \mathord{/} \mid \ldots
  & \text{binary operations}
\\
\expr & ::= &
  \var\x
\mid
  \local\x\expr\expr
  & \text{variable; sequencing}
\\&&
\mid
  \lit{c}
\mid
  \unop\uop\ux
\mid
  \binop\ux\bop\ux
  & \text{arithmetic operations of arity 0, 1, 2}
\\&&
\mid
  \utupleintro\xs
\mid
  \utupleelim\xs\expr\expr
  & \text{tuple construction and deconstruction}
\\&&
\mid
  f(\xs)
  & \text{function call}
\\
\tau & ::= &
  \real
  & \text{real numbers}
  \\&&
\mid
  (\vec\tau)
  & \text{tuples}
\\
\end{array}\]

%
A program is a sequence of function definitions. A function can take multiple
arguments and returns a single result. Functions are not recursive. The
language is statically typed. The types $\tau$ include the type $\real$ as well as
tuple types $(\vec\tau)$. See \file{Surface.ml} for more details.

A lexer and parser for Surface are provided
(\mll{SurfaceLexer}, \mly{SurfaceParser}).

A type-checker, an interpreter, and a pretty-printer for Surface are provided
(\mli{SurfaceTypeChecker}, \mli{SurfaceInterpreter}, \mli{SurfacePrinter}).

A~translation from Surface to Linear is provided (\mli{Surface2Linear}). This
transformation is trivial. A Surface variable $\x$ is transformed to an
unrestricted Linear variable $\ux$.

A random generator of well-typed Surface programs is provided
(\ml{SurfaceGenerator}). It is used in the test infrastructure.

% ------------------------------------------------------------------------------

\section{The Linear calculus}

Linear is also a calculus of arithmetic expressions.
%
In this calculus, there is a distinction between two classes of variables,
namely \emph{unrestricted variables~$\ux$} and \emph{linear variables~$\lx$}.

Furthermore, in Linear, an expression returns not just one result, but
\emph{a list of results}, which are further separated into \emph{a list of
  unrestricted results} and \emph{a list of linear results}. Each result is a
value, whose type can be $\real$ or a tuple type. We stress that a list of
results is not the same thing as a tuple: for instance, a function that
returns one unrestricted result of type $(\real, \real)$ is not the same thing
as a function that returns two unrestricted results of type $\real$ and
$\real$.

It helps to think of an expression visually as a \emph{box} with input and
output wires. Each wire is either unrestricted or linear. An input wire is
named: it corresponds to an unrestricted variable $\ux$ or a linear variable
$\lx$ that occurs free in the expression. An output wire is not named, but is
implicitly numbered. For instance, if an expression has two unrestricted
results and one linear result, then it has two unrestricted output wires
(implicitly numbered~0 and~1) and one linear output wire (implicitly
numbered~0).

These aspects are made precise by the type discipline presented in Figure~4 of
the paper. The source code of the Linear type-checker, in the file
\ml{LinearTypeChecker}, can also serve as a reference.

The syntax of Linear expressions is as follows. As a warm-up exercise, we
suggest drawing the boxes that correspond to each form of expression, as
well as their input and output wires.
%
\[\begin{array}{rcll}
\uop & ::= &
  \cos \mid \sin \mid \exp \mid \ldots
  & \text{unrestricted unary operations}
\\
\bop & ::= &
  \mathord{+} \mid \mathord{-} \mid \mathord{\times} \mid \mathord{/} \mid \ldots
  & \text{unrestricted binary operations}
\\
\lx & & & \text{linear variables} \\
\ux & & & \text{unrestricted variables} \\
\expr & ::= &
  \ret\uxs\lxs
\mid
  \seq\uxs\lxs\expr\expr
  & \text{return; sequencing}
\\&&
\mid
  \lit{c}
\mid
  \unop\uop\ux
\mid
  \binop\ux\bop\ux
  & \text{unrestricted fragment}
\\&&
\mid
  \lzero
\mid
  \ladd\lx\lx
\mid
  \lmul\ux\lx
\mid
  \drop\lx
\mid
  \dup\lx
  & \text{linear fragment}
\\&&
\mid
  \utupleintro\uxs
\mid
  \utupleelim\uxs\ux\expr
  & \text{unrestricted tuples}
\\&&
\mid
  \ltupleintro\lxs
\mid
  \ltupleelim\lxs\lx\expr
  & \text{linear tuples}
\\&&
\mid
  \funcall{f}\uxs\lxs
  & \text{function call}
\end{array}\]


As is evident in this syntax, linear variables take part in linear
computations only: the linear addition operator $\ladd\lx\lx$ takes two linear
variables as arguments, and the scaling operator $\lmul\ux\lx$ takes a linear
variables as its second argument. Addition and scaling are the only two
arithmetic operations that take linear variables as arguments. This ensures
that \emph{every expression denotes a linear function of its linear inputs to
  its linear outputs}.

Furthermore, in Linear, every linear variable must be used exactly once. This
is the reason why the expressions $\drop\lx$ and $\dup\lx$ are provided. The
expression $\drop\lx$ has one input wire, namely $\lx$, and zero output wire.
The expression $\dup\lx$ has one input wire, namely $\lx$, and two output
wires. It is usually known as a ``fan-out'' box, because a single wire is
split into two wires.

Thus, linear variables are ``linear'' both in the sense of mathematics
(linear variables serve as inputs of linear functions)
and in the sense of computer science
(linear variables are used exactly once).

Sometimes, ensuring that every linear variable is used exactly once can be
painful. We allow this discipline to be temporarily violated and repair it
afterwards (\mli{DupDropInsertion}). For example, forward-mode~AD is allowed
to produce code where this discipline is violated.

A type-checker, an interpreter, and a pretty-printer for Linear are provided
(\mli{LinearTypeChecker}, \mli{LinearInterpreter}, \mli{LinearPrinter}).

Two transformations that hoist and simplify $\kw{let}$ bindings are
provided (\mli{Normalize}, \mli{Simplify}). A transformation that renames
every variable to a fresh name is provided (\mli{Freshen}).

A transformation that transforms all linear variables into unrestricted
variables is provided (\mli{Forget}). It serves as a first step in the
translation of Linear back to Surface (\mli{Linear2Surface}).

Many useful auxiliary functions are provided in the file \ml{LinearHelp}.
It is worth spending some time to understand what these functions do; you
may need to use some or all of them.

% ------------------------------------------------------------------------------

\section{Forward-mode AD}

Task~1, the implementation of forward-mode AD, takes place in the file
\ml{ForwardMode}. The intended effect of this program transformation is
informally described in the file \mli{ForwardMode}. See also
\S\ref{sec:example:fmad} for an example.

This transformation is expected to produce code that computes a
Jacobian-Vector Product (JVP). The function \texttt{test\_forward\_mode}
in the file \ml{Test}, and the comment that precedes this function,
explain what property is expected.

An example is given in
\S\ref{sec:example:fmad}.

One way to test is via the following command:
\begin{verbatim}
  dune exec src/Main.exe -- --forward-mode --test-forward
\end{verbatim}
Or, you can edit \repofile{main.sh} to test just the forward mode
and then type \cmd{make test}.
More details about testing are given in \S\ref{sec:testing}.

% ------------------------------------------------------------------------------

\section{Reverse-mode AD}

Task~2, the implementation of reverse-mode AD, is split in two
transformations, unzipping and transposition, which take place
in the files \ml{Unzip} and \ml{Transpose}.

The composition of forward-mode AD, unzipping, and transposition is expected
to produce code that computes a Vector-Jacobian Product (VJP). The function
\texttt{test\_reverse\_mode} in the file \ml{Test}, and the comment that
precedes this function, explain what property is expected.

An example is given in
\S\ref{sec:example:rmad}.

One way to test is via the following command:
\begin{verbatim}
  dune exec src/Main.exe -- --reverse-mode --test-reverse
\end{verbatim}
Or, you can edit \repofile{main.sh} to test the reverse mode
and then type \cmd{make test}.
These commands test the \emph{combination} of forward-mode AD,
unzipping, and transposition, so they cannot be used until
all three passes have been implemented.
More details about testing are given in \S\ref{sec:testing}.

% [Gabriel] Je n'ai pas trouvé comment tester cette fonction
% (sans avoir fait la transposition encore), à part les vérifications
% de bonne formation qui sont faites, et lire le code généré sur des
% exemples.

% [Gabriel] La difficulté principale pour moi a été de trouver quel
% ensemble de variables trimballer à chaque endroit. L'article fait
% un produit explicite de tous les éléments du contexte, c'est moche
% et j'ai décidé de faire sans. (Visiblement François aussi, mais ce
% n'est pas explicité dans le sujet je crois ? juste dans les
% exemples.)

% [Gabriel] J'ai aussi oublié au départ d'inclure les paramètres de
% [uf] dans les paramètres de [lf] (en fait il suffit de mettre
% seulement ceux utilisés dans le corps de [lf], mais je n'en avais
% mis aucun). L'article n'était vraiment pas clair sur ce sujet, il
% fallait lire les petits caractères sur la définition de
% w (les parties du contexte *et de v*).
%
% Dans ma version j'ai décidé de ne pas inclure le sous-ensemble utile
% des paramètres d'entrée de [f] dans la sortie de [uf], comme fait
% François et l'article, mais de les passer directement de [ldf]
% à [lf]. Ça fait un poil moins de variables renvoyées par [uf], je
% trouve ça plus simple et plus joli. Ça déplace le placement de
% certains "dup"... j'espère que reverse-mode va marcher quand même ?

% [Gabriel]: dans Transpose, le point le plus délicat, de loin, est de
% comprendre comment permuter les variables dans les cas Ret, Let, LTupleIntro
% et FunCall.
%
% Je n'ai pas trouvé une façon de décrire ces permutations qui soit
% intuitive pour moi et me permettre savoir facilement où les placer
% et dans quel sens permuter.
%
% Je vois que la solution utilise [lwiden], j'ai défini mon propre
% opérateur de permutations.
%
% Je pense que le plus facile serait d'introduire des règles
% d'inférence plus algorithmiques qui représentent explicitement les
% permutations; on vérifie les règles et après ça se code tout
% seul. Je trouve la figure 11 de l'article, que j'ai utilisée comme
% référence, très légère sur ce point, elle cache complètement la
% difficulté sous le tapis.

% ------------------------------------------------------------------------------

\section{Example}
\label{sec:example}

\subsection{Forward-mode AD}
\label{sec:example:fmad}

As a simple example, let us consider the function $f : \mathbb{R}^2 \rightarrow
\mathbb{R}$ defined by $f(x_0, x_1) = x_0\times\cos(x_1)+x_0$.
%
This function is defined in Surface as follows:

\lstset{rangeprefix=(*\ ,rangesuffix=\ *),includerangemarker=false}
\begin{mdframed}[backgroundcolor=gray!10,linewidth=0pt]
\lstinputlisting[linerange=ORIG_SURFACE]{simple.txt}
\end{mdframed}

This program is translated down to Linear as follows:

\begin{mdframed}[backgroundcolor=gray!10,linewidth=0pt]
\lstinputlisting[linerange=ORIG_LINEAR]{simple.txt}
\end{mdframed}

Then, forward-mode AD (followed with insertion of $\kw{dup}$ and
$\kw{drop}$, normalization, simplification, freshening) produces the
following Linear program:

\begin{mdframed}[backgroundcolor=gray!10,linewidth=0pt]
\lstinputlisting[linerange=FMAD_LINEAR]{simple.txt}
\end{mdframed}

The function \texttt{f} has been preserved, and a new function \texttt{df},
which computes both \texttt{f} and the differential of \texttt{f}, has been
constructed.

By translating this Linear program back to Surface,
and by performing a compression step
(which transforms \oc|let x = e1 in e2| to \oc|e2[e1/x]|
 when there is at most one occurrence of \oc|x| in \oc|e2|),
one obtains the following code:

\begin{mdframed}[backgroundcolor=gray!10,linewidth=0pt]
\lstinputlisting[linerange=FMAD_RESURFACED]{simple.txt}
\end{mdframed}

It is not too difficult to check that the results computed by \texttt{df} are
correct: they are the original result of \texttt{f} and the differential of
\texttt{f}. This correctness property can be expressed in the following manner:
%
if \[[y, \dy] = \df(x_0, x_1, \dx_0, \dx_1)\]
then \[y=f(x_0, x_1)\]
and \[\dy=\Jf(x_0,x_1) \cdot \verticalvector{\dx_0}{\dx_1}\]
where
$\Jf(x_0, x_1)$ is the Jacobian matrix of $f$ at $(x_0, x_1)$.
Here, it is the matrix $[ \dd{f}{x_0}(x_0, x_1) \quad \dd{f}{x_1}(x_0, x_1) ]$.

It is worth noting that the intermediate result \texttt{hu6} is shared. Work
is never duplicated, so, up to a~constant factor, the cost of \texttt{df} is
the cost of \texttt{f}.

By design of the language Linear, a linear result can depend on an
unrestricted variable, but an unrestricted result cannot depend on a linear
variable. Thus, there is a one-way dependency between the unrestricted
fragment and the linear fragment of a computation. This is exploited during
unzipping.

\subsection{Reverse-mode AD}
\label{sec:example:rmad}

Now, let us start again from the Linear program that is produced by
forward-mode AD:

\begin{mdframed}[backgroundcolor=gray!10,linewidth=0pt]
\lstinputlisting[linerange=FMAD_LINEAR]{simple.txt}
\end{mdframed}

Unzipping transforms the function \texttt{df} into two functions, \texttt{udf}
and \texttt{ldf}, which represent the unrestricted fragment and the linear
fragment of the computation performed by \texttt{df}. Furthermore, the little
function \texttt{cdf} is a composition of \texttt{udf} and \texttt{ldf}. The
(normalized, simplified) Linear program produced by unzipping is as follows:

\begin{mdframed}[backgroundcolor=gray!10,linewidth=0pt]
\lstinputlisting[linerange=UNZIPPED_LINEAR]{simple.txt}
\end{mdframed}

The functions \texttt{ldf} and \texttt{cdf} have zero unrestricted outputs.
Thus, when their unrestricted inputs are fixed, these functions denote linear
maps of their linear inputs to their linear outputs.

These functions can be transposed. When an expression is viewed as a matrix,
this transformation corresponds to matrix transposition. When an expression is
viewed as a data flow network (composed of boxes and wires), this corresponds
to reversing the direction of the wires, transforming zero into
$\kw{drop}$ (and vice-versa), and transforming addition into
$\kw{dup}$ (and vice-versa).

The result is the following Linear program:

\begin{mdframed}[backgroundcolor=gray!10,linewidth=0pt]
\lstinputlisting[linerange=TRANSPOSED_LINEAR]{simple.txt}
\end{mdframed}

Once translated back to Surface (and compressed), the program is:

\begin{mdframed}[backgroundcolor=gray!10,linewidth=0pt]
\lstinputlisting[linerange=RMAD_RESURFACED]{simple.txt}
\end{mdframed}

The correctness of the function \texttt{tcdf} can be expressed as follows.
If
\([\dx_0, \dx_1] = \tcdf(x_0, x_1, \dy)\)
then
\[
  [\dx_0, \dx_1] =
  [ \dy ] \cdot \Jf(x_0, x_1)
\]
or, equivalently,
\[
  \verticalvector{\dx_0}{\dx_1} = \Jf(x_0, x_1)^\top \cdot [ \dy ].
\]

% ------------------------------------------------------------------------------

\section{Practical details}

\subsection{Building and using the compiler}
\label{sec:testing}

The directory \file{src/} contains the source files. The file \ml{Main}
contains the compiler's main program. Depending on the command line arguments,
this compiler can be applied either to a specific Surface program or to a
large number of randomly-generated Surface programs. Furthermore, it is
optionally capable of running tests to check that the transformed code seems
to be correct.

Typing \cmd{make} in the root directory produces the executable program
\texttt{\_build/default/src/Main.exe}. By default, this program expects to
receive on its command line the name of a file that contains a Surface
program.

The tiny script \repofile{main.sh} builds and runs this executable program. For
instance, typing \cmd{./main.sh test/inputs/simple.s} runs the compiler on the
Surface program stored in \repofile{test/inputs/simple.s}. You can edit this
script in order to add or remove some command line arguments. For instance, in
the beginning, you will want to remove \texttt{\dash test-forward} and
\texttt{\dash test-reverse}. You can restore them once you are ready to test
your implementations of forward-mode AD and reverse-mode AD.

Typing \cmd{./main.sh \dash help} shows the command line options accepted by the
compiler. In particular,
\begin{itemize}
\item \texttt{\dash forward-mode} causes the compiler to stop after performing
      forward-mode AD. You should use this option in the beginning and until
      your implementation of reverse-mode AD is ready.
\item \texttt{\dash reverse-mode} causes the compiler to perform
      forward-mode AD, unzipping, and transposition,
      thereby obtaining reverse-mode AD. It is enabled by default.
\item \texttt{\dash show-surface} causes the compiler to print the Surface
      program (on the standard output channel) at various stages of the
      transformation. This can be useful while debugging.
\item \texttt{\dash show-linear} causes the compiler to print the Linear
      program (on the standard output channel) at various stages of the
      transformation. This can be useful while debugging.
\item \texttt{\dash test-forward} causes the compiler to test the Surface
      program that is obtained as a result of forward-mode AD. The program is
      tested by checking that the equations given at the end
      of~\S\ref{sec:example:fmad} appear to hold. These equations
      are evaluated at randomly-chosen input values.
      All computations are performed using the
      arbitrary-precision real numbers provided by the library
      \href{https://github.com/backtracking/creal}{creal}.
      Because division is not defined (therefore not differentiable) at~0,
      if a test involves a division by a number that seems close to~0,
      then this test is aborted and ignored.
\item \texttt{\dash test-reverse} causes the compiler to test the Surface
      program that is obtained as a result of reverse-mode AD. The program is
      tested by checking that the equations given at the end
      of~\S\ref{sec:example:rmad} appear to hold.
\end{itemize}

When invoked without a file name, the compiler automatically applies itself to
a number of randomly-generated Surface programs of increasing sizes, up to a
certain size, which is specified on the command line via \texttt{\dash budget
  <int>}. The command \cmd{make auto} exploits this to perform random tests up
to size~100.

The command \cmd{make human} applies the compiler to each of the Surface
programs stored in the directory \repofile{test/inputs/}. This directory
currently contains only a few Surface programs. You are encouraged to write
more. Writing several very small Surface programs that exercise various
features, one at a time, should help you debug your compiler.

The command \cmd{make test} combines \cmd{make human} and \cmd{make auto}.

\textbf{Task~1 is to complete the file \ml{ForwardMode} so that \cmd{make
    test} succeeds, with \texttt{\dash forward-mode} and \texttt{\dash
    test-forward} enabled.}

\textbf{Task~2 is to additionally complete the files \ml{Unzip} and
  \ml{Transpose} so that \cmd{make test} succeeds, with \texttt{\dash
    reverse-mode} and \texttt{\dash test-forward} and \texttt{\dash
    test-reverse} enabled.}

% ------------------------------------------------------------------------------

\section{Required software}

To use the sources that we provide, you need OCaml 4.14 and Menhir.
You also need the library \texttt{pprint}.
Assuming that you have installed OCaml via \texttt{opam},
these components can be installed by typing
\cmd{opam update \&\& opam install menhir pprint}.

% ------------------------------------------------------------------------------

\section{Evaluation}

Assignments will be evaluated by a combination of:
\begin{itemize}
\item \textbf{Testing}. Your compiler will be tested with the input programs
  that we provide (make sure that \cmd{make test} succeeds!) and with
  additional input programs.
\item \textbf{Reading}. We will browse through your source code and evaluate
  its correctness and elegance.
\end{itemize}

The \textbf{correctness} of your code matters;
its performance does not.
It is acceptable to favor clarity over efficiency:
for instance,
it is permitted to call the functions \texttt{flv} or \texttt{fuv}
(which compute the free variables of a term)
without worrying about the cost of these calls,
even if (as a result of these calls)
the complexity of a program transformation becomes
quadratic instead of linear.

% ------------------------------------------------------------------------------

\section{Extra credit}

For extra credit, or just for fun, you may wish to go beyond what is strictly
requested. What to do is up to you. Here are some suggestions of things to do.
Beware: we do not know exactly how difficult or time-consuming these
extensions are.
%
\begin{itemize}
\item Extend Surface and Linear with more primitive arithmetic operations.
\item Implement a measure of the runtime cost of an expression,
      and check (experimentally) that each transformation preserves the cost
      (perhaps up to certain constant factor),
      as claimed in Radul \etal.'s paper.
\item Test each of the three main transformations
      (\fmad; \unzipping; \transposition) in isolation.
      This requires a generator of Linear programs.
\item At the level of the Surface language,
      perform algebraic simplifications by identifying
      and simplifying expressions such as \oc|0 + e|,
      \oc|0 * e|, \oc|1 * e|, and so on.
\item At the level of the Surface language,
      implement a form of common subexpression elimination (CSE).
\end{itemize}

% ------------------------------------------------------------------------------

\section{What to send}

When you are done, please \href{mailto:francois.pottier@inria.fr}{send to
  François Pottier} a \file{.tar.gz} archive containing your completed
programming project. The archive should contain a single
directory \texttt{mpri-2.4-projet-2022-2023}.

Please include a file \textbf{README.md} to describe what you have achieved
(task~1; task~2; extra tasks). You are welcome to provide explanations (in
French or in English) about your solution or about the difficulties that you
have encountered.

% ------------------------------------------------------------------------------

\section{Deadline}

Please send your project on or before
\textbf{Monday, February 13, 2023}.
% à peu près 3 semaines avant l'examen final prévu le 8 mars

% ------------------------------------------------------------------------------

\bibliography{local}

\end{document}
